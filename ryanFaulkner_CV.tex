%%%%%%%%%%%%%%%%%%%%%%%%%%%%%%%%%%%%%%%%%%%%%%%%%%%%%%%%%%%%%%%%%%%
%% 
%% Ryan Faulkner's resume
%%   - based off work by Michael DeCorte 
%%
%%%%%%%%%%%%%%%%%%%%%%%%%%%%%%%%%%%%%%%%%%%%%%%%%%%%%%%%%%%%%%%%%%%



%%
%% The following code sets up the document formatting
%%

%this assumes that res_yy.sty is in some path
\documentstyle[hyperref, margin, line]{res_yy}

\hypersetup{backref,pdfpagemode=Full,colorlinks=true,backref}

\addtolength{\oddsidemargin}{-0.45in}
\addtolength{\voffset}{-0.30in}
\addtolength{\textwidth}{1.00in} \addtolength{\textheight}{1.50in}

\renewcommand{\namefont}{\LARGE\emph}


%%
%% The following code defines some macros for terms which have raised font
%% (ie 4\fourth would result 4th with the 'th' raised (superscripted)
%%

\def\Cplusplus{{\rm C\raise.5ex\hbox{\small ++}}}
\def\CSharp{{\rm C\raise.5ex\hbox{\small \#}}}
% 'st' 'nd' 'rd' 'th' superscripts for numbers
\def\first{{\raise.5ex\hbox{\small st}}}
\def\second{{\raise.5ex\hbox{\small nd}}}
\def\third{{\raise.5ex\hbox{\small rd}}}
\def\fourth{{\raise.5ex\hbox{\small th}}}



%%
%% starting the actual document
%%

\begin{document}

%the name in big fonts at the top of resume
%this is left aligned
\name{Ryan Faulkner}

%this is right aligned
\address{
 phone: (415) 796-5086 \ \ \ \ \ \ \ \ \ \ website: http://www.cs.mcgill.ca/\textasciitilde rfaulk \ \ \ \ \ \ \ \ \ \ email: ryan.faulkner@mail.mcgill.ca
}

\begin{resume}

\section{\textsc{Qualifications}}

\emph{Graduate Level Courses Completed}: Bayesian Statistics, Probabilistic Analysis of Algorithms, Machine Learning, Neural Networks, Fundamentals of Computer Vision, Statistical Computer Vision, Natural Language Processing, Computer Graphics, Matrix Computation, Formal Languages and Automata   

\emph{Programming Languages}: Python, PHP, \Cplusplus, Java, \CSharp, Javascript, MATLAB, R, Perl, Assembly (Intel and Motorola Instruction Sets)

\emph{Frameworks}: Django, Flask, JQuery, Boost, smarty

\emph{Operating Systems, Standards}: Unix, Linux, Windows, HTTP, TCP/IP, JSON, XML, REST, CSS, HTML, Map Reduce, \LaTeX 

\emph{Technologies}: Git, SVN, MySQL, Apache: [HTTP Server, Hadoop, Hive, Pig, Oozie, HBase, Flume, Avro], Kafka, Neo4j


%%
%% the meat of the resume starts now
%%

\begin{formatb}
  \employer{l}\title{r}\\
  \location{l}\dates{r}\\   
  \body\\
\end{formatb}

\section{\textsc{Experience}}

\employer{\textbf{Flickr (Yahoo! Inc.)}}
\title{Data Engineer}
\location{San Francisco, CA}
\dates{Spring 2013 - Present}
\begin{position}
\vspace{-8pt}
\begin{itemize}
  \item Low level instrumentation of all Flickr web \& API traffic (PHP, Apache)
  \item Data modeling, storage architecture, visualization and reporting (MySQL, Redis, PHP, Javascript, Hadoop, Cloud storage, Offline Job Processing)
  \item Design and implementation of pipeline architecture implemented via streaming map-reduce and high volume writes to cloud key-value store for analytics (Hadoop, Map-reduce, Oozie, Hive, Pig, Python, Cloud Storage)
  \item Growth team, strategy, development and reporting around metrics critical to Flickr's growth
  \item Growth \& Data related feature development (PHP, Javascript)
\end{itemize}
\vspace{4pt}
\end{position}

\employer{\textbf{Wikimedia Foundation}}
\title{Research Analyst}
\location{San Francisco, CA}
\dates{Fall 2010 - Spring 2013}
\begin{position}
\vspace{-8pt}
\begin{itemize}
  \item  Built end to end systems to support analytics for Wikimedia 2010 and 2011 online annual Fundraisers
  \item  Performed in-depth analyses including statistical modeling of donor data and web traffic from 2010 / 2011 Fundraisers
  \item  Analytics over the Wikipedia editor community data including editing trends over the history of Wikipedia and to establish an understanding of editor behaviour for growth
  \item  Built RESTful API (Python/Flask) to provide a means of generating high level metrics over user cohorts at scale
\end{itemize}
\vspace{4pt}
\end{position}


\employer{\textbf{McGill University}}
\title{Instructor}
\location{Montreal, QC}
\dates{Winter 2010}
\begin{position}
Computers in Engineering (COMP-208):  Course topics included introduction to programming in C and Fortran, numerical methods for function approximation, root finding, and integral estimation, pointers, arrays, sorting and searching.  Held two lectures per week with office hours.
\end{position}

\employer{\textbf{McGill University}}
\title{Teaching Assistant}
\location{Montreal, QC}
\dates{Fall 2009}
\begin{position}
Data Structures and Algorithms (COMP-251):  Course topics included linked lists, stacks, queues, heaps, trees, divide and conquer, greedy algorithms, dynamic programming, graph search, network flow, and asymptotic analysis of running time.  Marked assignments and tests, and held office hours weekly.
\end{position}

\newpage

\employer{\textbf{INFOR Corporation - http://www.infor.com/}}
\title{Software Consultant}
\location{Toronto, ON}
\dates{Fall 2005 - Summer 2008}
\begin{position}
\vspace{-8pt}
\begin{itemize}
  \item  Advised corporate clients on integrating our workforce management software
  \item  Interacted closely with technical teams (management, developers, database administrators)
  \item  On-site customization of the product to meet requirements (Java, JSP, Javascript and a strong emphasis on design patterns and reusable solutions)
\end{itemize}
\vspace{4pt}
\end{position}


% add a blank line for layout

%%
%% We use the same formatting for projects as for work experience
%% Shown below is the formatting used previously
%%
%%  \begin{formatb}
%%    \employer{l}\title{r}\\
%%    \location{l}\dates{r}\\
%%    \body\\
%%  \end{formatb}
%%
%% 
%%  Note that \location is now being used for non-location information
%%


%\section{\textsc{Research Interest}}
%M

\section{\textsc{Education}}

\textbf{McGill University} \hfill 2008 - 2010 \\
MSc. Student in Computer Science, CGPA: 3.9 / 4.0 \hfill Advisor: Doina Precup \\
\newline
\textbf{University of Melbourne} \hfill 2004 \\ 
ISXO (University of Toronto) Exchange Program \\
\newline
\textbf{University of Toronto} \hfill 2000 - 2005 \\ 
Bachelor of Applied Science - Computer Engineering, CGPA: 3.0 / 4.0


\begin{formatb}
  \employer{l}\dates{r}\\
  \body\\
\end{formatb}

\section{\textsc{Research}}

My Masters research involved using Deep Belief Networks as generative simulators for environments in the context of Reinforcement Learning domains.  The architecture of this approach, termed \emph{Dyna}, was formally introduced by Richard Sutton in \emph{First results with Dyna, an integrated architecture for learning, planning, and reacting} (1990).  I am also interested in investigating new ways of training deep models and neural networks which lead to improvement on widely benchmarked datasets in Machine Learning, applications in Computer Vision, and Bayesian Learning.  More information can be found on my website.

\employer{\textbf{Dyna Planning using a Feature Based Generative Model}}
\dates{December 2010}
\begin{position}
Neural Information Processing Systems (NIPS) \\
Poster - Deep Learning and Unsupervised Feature Learning Workshop \\
Authors: Faulkner, Ryan; Precup, Doina
\end{position}

\employer{\textbf{Dyna Learning with Deep Belief Networks}}
\dates{August 2010}
\begin{position}
MSc. Thesis, McGill University \itshape{(submitted - under review)} \\
Authors: Faulkner, Ryan
\end{position}

\employer{\textbf{Deep Belief Networks in Reinforcement Learning}}
\dates{March 2010}
\begin{position}
Reasoning and Learning Lab Meeting Presentation Topic, McGill University \\
Authors: Faulkner, Ryan
\end{position}

\section{\textsc{Projects}}

See my github page: https://github.com/rfaulkner

\employer{\textbf{Flickipedia (http://www.flikiwikimash.com)}}
\dates{Spring 2014}
\begin{position}
Authors: Faulkner, Ryan; \\
Originally a hackday project at Yahoo this is a mashup between Flickr and Wikipedia.
\end{position}

\employer{\textbf{Melody Prediction with HMMs under Bayesian Learning}}
\dates{Spring 2010}
\begin{position}
Authors: Faulkner, Ryan; Siddiqui, Mansoor \\
Learning a generative model to extract the high level features of music data.
\end{position}

\employer{\textbf{Visual Learning for Object Recognition}}
\dates{Spring 2009}
\begin{position}
Authors: Faulkner, Ryan \\
Object recognition using multivariate parametric models over a reduced state space. 
\end{position}

\employer{\textbf{A Robust Approach to 3D Object Recognition}}
\dates{Fall 2008}
\begin{position}
Authors: Faulkner, Ryan; Siddiqui, Mansoor \\
Principal Components Analysis applied to the COIL-100 dataset to train a model for object recognition that is robust to pose and illumination variance in the data.
\end{position}


%%
%% This section could also use more formatting, but looks ok, as is
%%


%%\section{\textsc{Awards}}

%%\textbf{McGill Provost Graduate Award}; 2008

%%\textbf{Canada Millennium Scholarship}; 2004

%%\textbf{University of Toronto Scholar}; 2000

%%\textbf{Governor General's Academic Medal} for highest OAC (Ontario Highschool) Average; 2000

%%\textbf{T.W. Martin Scholarship} for excellence in Mathematics; 2000

\section{\textsc{References}}

Available upon request.

%%
%% Note that we're redefining the formatting
%% We only have one row of information now, instead of two
%%


%%
%% Nothing special here, just a normal table
%%

%\section{\textsc{Course Work}}
%  \begin{tabular}{lllll}
%  Information Networks   & \ \ & Machine Learning    & \ \ & Theory of Computation \\ 
%  Computer Graphics      & \ \ & Machine Vision      & \ \ & Programming Languages \\
%  Software Engineering   & \ \ & Algorithms          & \ \ & Artificial Intelligence     \\
%  Operating Systems      & \ \ & Databases           & \ \ & Computer Architecture \\
%  Numerical Methods      & \ \ & Graph Theory        & \ \ & Differential Equations      \\
%  Probability Theory     & \ \ & Number Theory       & \ \ & Differential Geometry       \\
%  Advanced Calculus      & \ \ & Abstract Algebra    & \ \ & Advanced Combinatorics   \\
%  \end{tabular}


\end{resume}
\end{document}
